\documentclass[12pt]{article}

\usepackage{sbc-template}

\usepackage{graphicx,url}

\usepackage[brazil]{babel}   
%\usepackage[latin1]{inputenc}  
\usepackage[utf8]{inputenc}  
% UTF-8 encoding is recommended by ShareLaTex


\usepackage{amsthm}
\theoremstyle{plain}

\newtheorem{theorem}{Theorem}
\newtheorem*{theorem*}{Definição}     
\sloppy

\title{Estratégias de coreografia na arquitetura de microserviços: Um estudo exploratório}

\author{Marcelo de Rezende Martins\inst{1}, Mauricio Ribeiro\inst{1}, John Sql\inst{1} }


\address{Instituto de Pesquisas Tecnológicas -
  (IPT)\\
  São Paulo -- SP -- Brazil
  \email{rezende.martins@gmail.com, eng.mauricio@outlook.com,
  john.sql@gmail.com}
}

\begin{document} 

\maketitle

\begin{abstract}
  This meta-paper describes the style to be used in articles and short papers
  for SBC conferences. For papers in English, you should add just an abstract
  while for the papers in Portuguese, we also ask for an abstract in
  Portuguese (``resumo''). In both cases, abstracts should not have more than
  10 lines and must be in the first page of the paper.
\end{abstract}
     
\begin{resumo} 
  Com a crescente utilização de microserviços na arquitetura de software também foi necessário desenvolver a técnica de interação entre os serviços. Uma técnica para coordenar a troca de mensagens entre os serviços é a coreografia. A técnica de coreografia, já conhecida nas arquitetura SOA (através da WS-CDL, por exemplo), é amplamente utilizada na arquitetura de microserviços. Dado a sua importância, este artigo busca identificar quais são as diferentes estratégias de coreografia através de um estudo exploratório. 
\end{resumo}


\section{Introdução}
\label{sec:introducao}

A arquitetura baseada em microserviços tornou-se muito popular nos dias de hoje. Muitas empresas estão migrando suas aplicações SOA e/ou monolíticos para uma arquitetura mais flexível como microserviços. O advento do microserviço deve-se a vários avanços tecnológicos nos últimos anos. Entre eles estão:
\begin{itemize}
    \item Arquitetura REST
    \item Formato JSON
    \item DDD (Domain Driven Design)
    \item Cloud Computing
    \item IAAS (Infrastructure As A Service)
    \item PaaS (Platform As A Service)
\end{itemize}

Do ponto de vista técnico, os microserviços devem ser componentes independentes conceitualmente, implantados isoladamente e equipados com recursos dedicados como memória e ferramentas de persistência \cite{Dragoni2017}.

E segundo \cite{Dragoni2017}, uma arquitetura de microserviços é uma aplicação distribuída onde todos os módulos são microserviços.

Umas das justificativas para adoção da arquitetura de microserviços por parte das empresas é a facilidade na hora de escalar. Enquanto numa aplicação monolítica, o custo para escala é alto, pois você deve implantar uma aplicação completa numa outra máquina (escala horizontal), já na microserviços, você implanta somente o serviço que está sendo mais requisitado. Esta é uma vantagem da arquitetura de microserviços, pois você consegue aumentar os recursos somente para a parte com maior \textit{workload} do sistema. 

Para \cite{wolf:2018}, a principal vantagem dos microserviços é o baixo acoplamento e alta coesão. O baixo acoplamento e a alta coesão facilita a divisão do desenvolvimento entre várias equipes. Além disso, num mercado altamente competitivo, a arquitetura de microserviços facilita a implantação de novas funcionalidades. A manutenção é facilitada, pois basta alterar o microserviço com problema, sem a necessidade de fazer a implantação do sistema inteiro.

Segundo \cite{jung:2017},as arquiteturas de microserviços compartilham algumas caraterísticas, dentre elas:

\begin{itemize}
    \item \textbf{Descentralizado}: Arquitetura de microserviços são sistemas distribuídos com o gerenciamento de dados descentralizados. Conforme citado anteriormente, cada microserviço contém sua própria visão do modelo de dados. E o microserviço também é descentralizado no modo como ele é desenvolvido, implantado, gerenciado e operacionalizado.
    \item \textbf{Independente}: Diferentes componentes podem ser alterados, atualizados ou substituídos sem afetar a funcionalidade de outros componentes. E os times responsáveis por cada microserviço podem agir de forma independente.
    \item \textbf{\textit{Do one thing well}}: Cada microserviço deve ser projetado para um conjunto de funcionalidades e com foco num determinado domínio específico.
    \item \textbf{Caixa-preta}: Os componentes são projetados como caixa-preta, isto é, eles escondem os detalhes da sua complexidade dos outros componentes. A comunicação deve ser feita através de uma API bem definida para evitar dependências implícitas e/ou escondidas.
\end{itemize}

E de acordo com \cite{jung:2017}, os principais benefícios apontados por quem adota esta arquitetura são:
\begin{itemize}
    \item \textbf{Agilidade}: O desenvolvimento é dividido entre times pequenos, cada uma responsável pelos seus serviços. Portanto, a equipe atua num contexto delimitado e bem definido, facilitando a compreensão e reduzindo o ciclo de desenvolvimento. 
    \item \textbf{Inovação}: Com a independência dos microserviços, a alteração e o teste de novas idéias são facilitadas. O baixo custo para implantar um novo serviço, facilita a criação de uma cultura de mudança e inovação.
    \item \textbf{Qualidade}: Os benefícios da divisão do desenvolvimento do software em componentes menores e bem definidos são similares aos benefícios vistos na programação orientada a objetos: reusabilidade, composibilidade e manutenibilidade.
    \item \textbf{Escalabilidade}: Com o baixo acomplamento e alta coesão, os serviços podem ser escalados horizontalmente e independente de cada um. Além disso, a resiliência da aplicação pode ser improvisada, pois os serviços são facilmente substituídos.
\end{itemize}

Porém, a adoção da arquitetura de microserviços envolve vários desafios. Existe, por exemplo, a \textit{falácia da computação distribuída}, no qual os novos programadores na arquitetura distribuída acreditam que a comunicação na rede é confiável, a latência é zero e a banda de rede é infinita. Há também a dificuldade na migração de um sistema monolítico para uma arquitetura de microserviços. Para realizar esta tarefa, é necessário definir como o sistema será dividido, quais módulos serão criados, o escopo de cada módulo. E lembrando que os serviços devem ser independentes, coesos \cite{jung:2017}. 

Através de um estudo sistemático, \cite{Alshuqayran:2016} identificou os principais desafios enfrentados na arquitetura de microserviços. Os principais desafios encontrados por \cite{Alshuqayran:2016} foram:
\begin{itemize}
    \item Integração/Comunicação
    \item \textit{Service Discovery}
    \item Desempenho
    \item Tolerância a falhas
    \item Segurança
    \item \textit{Tracing e Logging}
    \item Monitoramento do desempenho da aplicação
    \item Operação de implantação
\end{itemize}

Logo, a adoção da arquitetura de microserviços não é uma tarefa simples. Há problemas desde integração e comunicação dos servicos, segurança e até organizacional, como definir a equipe de desenvolvimento e como mudar a cultura da empresa para seguir um método \textit{DevOps}, por exemplo \cite{jung:2017}. Um ponto importante e um desafio na arquitetura de microserviços é como manter o baixo acomplamento e alta coesão. Conforme o sistema cresce e mais lógicas complexas são acrescentadas, surge a necessidade de gerenciar como é feita a integração entre os serviços. 

Segundo \cite{Dragoni2017}, microserviço é um processo independente coeso interagindo através de mensagens. E \cite{Newman:15}, sugere que esta troca de mensagens deva ser feita via \textit{REST}. E há dois estilos arquiteturais que podem ser seguidos para coordenar esta troca de mensagens. Uma é a orquestração, no qual a coordenação dos múltiplos serviços é feita através de um mediador central ou um \textit{hub} integrador (Mule, Camel, Spring Integration etc.). E a outra é a coreografia, no qual a coordenação das múltiplas chamadas dos serviços é feita sem um mediador central \cite{richards:15, Newman:15}. 

Em geral, segundo \cite{Newman:15}, os sistemas que utilizam coreografia para coordenar a troca de mensagens tendem a ter um menor acoplamento, são mais flexíveis e aptos a mudanças. E boa parte dos autores \cite{wolf:2018, Dragoni2017, richards:15, Alshuqayran:2016} seguem a mesma linha de \cite{Newman:15} e sugerem que na arquitetura de microserviços a coreografia seja preterida em favor da orquestração.

\section{Objetivo} \label{sec:firstpage}

O objetivo deste artigo é fazer um estudo exploratório para investigar quais são as estratégias de coreografia utilizadas na arquitetura de microserviços. E também tentar identificar quais são as ferramentas comumente utilizadas. Na seção \ref{sec:trabalhos-relacionados} é apresentado os trabalhos que fazem referência a arquitetura de microserviços e o uso de coreografia. A seção \ref{sec:resultados} apresenta a compilação das estratégias de coreografia e ferramentas comumente citadas na literatura e/ou utilizados pela mercado. E a seção \ref{sec:conclusao} discute o resultado do artigo e trabalhos futuros.

\section{Trabalhos relacionados}
\label{sec:trabalhos-relacionados}

Segundo \cite{wolf:2018}, um dos benefícios da arquitetura de microserviços é o baixo acoplamento e alta coesão do sistema. Para \cite{Newman:15}, sistemas que utilizam coreografia tendem a ter um menor acoplamento e são de fácil manutenção. A coreografia é um estilo arquitetural, é uma forma de definir como os serviços interagem entre si. No caso da coreografia, os microserviços interagem entre si sem um mediador central. A luz disso, uma possível pergunta é:

\emph{Como coreografar os microserviços para que o sistema tenha um baixo acoplamento e alta coesão?}

Para \cite{damore:2018}, é possível obter o baixo acoplamento utilizando uma coreografia baseado em eventos. Neste caso, os microserviços 



\section{Resultados}
\label{sec:resultados}

Seção de resultados

\section{Conclusão}
\label{sec:conclusao}

Bibliographic references must be unambiguous and uniform.  We recommend giving
the author names references in brackets, e.g. \cite{knuth:84},
\cite{boulic:91}, and \cite{smith:99}.

The references must be listed using 12 point font size, with 6 points of space
before each reference. The first line of each reference should not be
indented, while the subsequent should be indented by 0.5 cm.

\bibliographystyle{sbc}
\bibliography{sbc-template}

\end{document}
