\documentclass[12pt]{article}

\usepackage{sbc-template}

\usepackage{graphicx,url}

\usepackage[brazil]{babel}   
%\usepackage[latin1]{inputenc}  
\usepackage[utf8]{inputenc}  
% UTF-8 encoding is recommended by ShareLaTex


\usepackage{amsthm}
\theoremstyle{plain}

\newtheorem{theorem}{Theorem}
\newtheorem*{theorem*}{Definição}     
\sloppy

\title{Estratégias de coreografia na arquitetura de microserviços: Uma pesquisa exploratória}

\author{Marcelo de Rezende Martins\inst{1}, Mauricio Ribeiro\inst{1}, John Alves de Medeiros Silva\inst{1} }


\address{Instituto de Pesquisas Tecnológicas -
  (IPT)\\
  São Paulo -- SP -- Brazil
  \email{rezende.martins@gmail.com, eng.mauricio@outlook.com,
  john.sql@gmail.com}
}

\begin{document} 

\maketitle

\begin{abstract}
  This meta-paper describes the style to be used in articles and short papers
  for SBC conferences. For papers in English, you should add just an abstract
  while for the papers in Portuguese, we also ask for an abstract in
  Portuguese (``resumo''). In both cases, abstracts should not have more than
  10 lines and must be in the first page of the paper.
\end{abstract}
     
\begin{resumo} 
  Com a crescente utilização de microserviços na arquitetura de software também foi necessário desenvolver a técnica de interação entre os serviços. Uma técnica para coordenar a troca de mensagens entre os serviços é a coreografia. A técnica de coreografia, já conhecida nas arquitetura SOA \footnote{Utilizamos o mesmo significado para o termo SOA que o \cite{martinfowler-microservices:2014} adotaram.} (através da WS-CDL, por exemplo), é amplamente utilizada na arquitetura de microserviços. Dado a sua importância, este artigo busca identificar quais são as diferentes estratégias de coreografia através de uma pesquisa exploratória. 
\end{resumo}


\section{Introdução}
\label{sec:introducao}

A arquitetura baseada em microserviços tornou-se muito popular nos dias de hoje. Muitas empresas estão migrando suas aplicações SOA \footnotemark[1] e/ou monolíticos para uma arquitetura mais flexível como microserviços. O advento do microserviço deve-se a vários avanços tecnológicos nos últimos anos. Para \cite{Newman:15}, os principais foram:
\begin{itemize}
    \item \textit{Domain-Driven Design}
    \item Entrega contínua (\textit{Continuous Delivery})
    \item Virtualização por demanda (\textit{On-demand virtualization})
    \item Automação da infra-estrutura
    \item Times pequenos autônomos
    \item Sistemas escaláveis
\end{itemize}

Segundo \cite{Newman:15}, microserviço é um serviço autônomo e pequeno. E \cite{martinfowler-microservices:2014} complementa, microserviço é um serviço que é executado no seu próprio processo e utiliza mecanismos leves de comunicação. Normalmente esta comunicação é feita por meio de uma API (\textit{Application Programming Interface} utilizando o protocolo HTTP (\textit{Hypertext Transfer Protocol}).

E segundo \cite{Dragoni2017, martinfowler-microservices:2014}, uma arquitetura de microserviços é uma aplicação distribuída onde todos os módulos são microserviços.

Umas das justificativas para adoção da arquitetura de microserviços por parte das empresas é a facilidade na hora de escalar. Enquanto numa aplicação monolítica, o custo para escala é alto, pois você deve implantar uma aplicação completa numa outra máquina (escala horizontal), já na microserviços, você implanta somente o serviço que está sendo mais requisitado. Esta é uma vantagem da arquitetura de microserviços, pois você consegue aumentar os recursos somente para a parte com maior \textit{workload} do sistema. 

Para \cite{wolf:2018}, a principal vantagem dos microserviços é o baixo acoplamento e alta coesão. O baixo acoplamento e a alta coesão facilita a divisão do desenvolvimento entre várias equipes. Além disso, num mercado altamente competitivo, a arquitetura de microserviços facilita a implantação de novas funcionalidades. A manutenção é facilitada, pois basta alterar o microserviço com problema, sem a necessidade de fazer a implantação do sistema inteiro.

Segundo \cite{martinfowler-microservices:2014},as arquiteturas de microserviços compartilham algumas caraterísticas, dentre elas:

\begin{itemize}
    \item Componentização através de serviços
    \item Organizado em torno da capacidade do negócio
    \item Produto não Projeto
    \item Serviços inteligentes e intermediários simples (\textit{Smart endpoints and dumb pipes})
    \item Governança descentralizada
    \item Gerenciamento de dados descentralizado
    \item Automação da infra-estrutura
    \item Tolerância a falhas
    \item Projeto evolutivo
\end{itemize}

E de acordo com \cite{Alshuqayran:2016}, os principais benefícios apontados por quem adota esta arquitetura são o aumento em:
\begin{itemize}
    \item Agilidade
    \item Produtividade do desenvolvedor
    \item Resiliência
    \item Escalabilidade
    \item Confiabilidade
    \item Manutenibilidade
    \item Fácil implantação
\end{itemize}

Porém, a adoção da arquitetura de microserviços envolve vários desafios. Existe, por exemplo, a \textit{falácia da computação distribuída}, no qual os novos programadores na arquitetura distribuída acreditam que a comunicação na rede é confiável, a latência é zero e a banda de rede é infinita. Há também a dificuldade na migração de um sistema monolítico para uma arquitetura de microserviços. Para realizar esta tarefa, é necessário definir como o sistema será dividido, quais módulos serão criados, o escopo de cada módulo. E lembrando que os serviços devem ser independentes, coesos \cite{jung:2017}. 

Através de um estudo sistemático, \cite{Alshuqayran:2016} identificou os principais desafios enfrentados na arquitetura de microserviços. Os principais desafios encontrados por \cite{Alshuqayran:2016} foram:
\begin{itemize}
    \item Integração/Comunicação
    \item \textit{Service Discovery}
    \item Performance
    \item Tolerância a falhas
    \item Segurança
    \item \textit{Tracing e Logging}
    \item Monitoramento do desempenho da aplicação
    \item Operação de implantação
\end{itemize}

Logo, a adoção da arquitetura de microserviços não é uma tarefa simples. Há problemas desde integração e comunicação dos servicos, segurança e até organizacional, como definir a equipe de desenvolvimento e como mudar a cultura da empresa para seguir um método \textit{DevOps}, por exemplo \cite{jung:2017}. Um ponto importante e um desafio na arquitetura de microserviços é como manter o baixo acomplamento e alta coesão. Conforme o sistema cresce e mais lógicas complexas são acrescentadas, surge a necessidade de gerenciar como é feita a integração entre os serviços. 

 Para \cite{Newman:15}, há dois estilos arquiteturais que podem ser seguidos. Uma é a orquestração, no qual a coordenação dos múltiplos serviços é feita através de um mediador central ou um \textit{hub} integrador (Mule, Camel, Spring Integration etc.). E a outra é a coreografia, no qual a coordenação das múltiplas chamadas dos serviços é feita sem um mediador central \cite{richards:15, Newman:15}. 

Em geral, segundo \cite{Newman:15}, os sistemas que utilizam coreografia para coordenar a troca de mensagens tendem a ter um menor acoplamento, são mais flexíveis e aptos a mudanças. E boa parte dos autores \cite{wolf:2018, Dragoni2017, richards:15, Alshuqayran:2016, martinfowler-microservices:2014} seguem a mesma linha de \cite{Newman:15} e sugerem que na arquitetura de microserviços a coreografia seja preterida em favor da orquestração.

\section{Objetivo} \label{sec:firstpage}

De acordo com \cite{gil:17}, o objetivo de uma pesquisa exploratória é:

"\emph{Estas pesquisas têm como objetivo proporcionar maior familiaridade com o problema com vistas a
tomá-lo mais explícito ou a construir hipóteses. Pode-se dizer que estas pesquisas têm como
objetivo principal o aprimoramento de idéias ou a descoberta de intuições.}"

O objetivo deste artigo é ampliar e aprimorar o conhecimento sobre as diferentes estratégias de coreografia utilizadas na arquitetura de microserviços. Na seção \ref{sec:trabalhos-relacionados} é apresentado os trabalhos que fazem referência a arquitetura de microserviços e o uso de coreografia. A seção \ref{sec:resultados} apresenta a compilação das estratégias de coreografia e ferramentas comumente citadas na literatura e/ou utilizados pela mercado. E a seção \ref{sec:conclusao} discute o resultado do artigo e trabalhos futuros.

\section{Trabalhos relacionados}
\label{sec:trabalhos-relacionados}

Segundo \cite{wolf:2018}, um dos benefícios da arquitetura de microserviços é o baixo acoplamento e alta coesão do sistema. Para \cite{Newman:15}, sistemas que utilizam coreografia tendem a ter um menor acoplamento e são de fácil manutenção. A coreografia é um estilo arquitetural, é uma forma de definir como os serviços interagem entre si. No caso da coreografia, os microserviços interagem entre si sem um mediador central. A luz disso, uma possível pergunta é:

\emph{Como coreografar os microserviços para que o sistema tenha um baixo acoplamento e alta coesão?}

Para \cite{damore:2018, Newman:15}, é possível obter o baixo acoplamento utilizando uma colaboração baseada em eventos. Os microserviços conectam-se a um barramento de eventos (\textit{event bus}), e quando uma ação ocorre, o serviço envia uma mensagem indicando a ocorrência do evento. Esta mensagem será enviada ao barramento, que notificará os serviços interessados (assinantes). Este modelo orientado a eventos utiliza a troca de mensagens assíncrona como forma de comunicação. Neste caso, é necessário um intermediário nesta troca de mensagens (\textit{message broker}). 

Segundo \cite{Curry2004}, um software e/ou infra-estrutura que provê o suporte e a capacidade necessária para a troca de mensagens entre os componentes de um sistema distribuído, é chamado de \textit{Message Oriented Middleware} (MOM). Nesta infra-estrutura, cada componente e/ou serviço conecta-se a uma camada intermediária (\textit{message broker}), que é responsável pelo recebimento e envio de mensagens. Os sistemas baseados no MOM não tem problema de espera (ou bloqueio), pois os serviços podem enviar as mensagens que são persistidas no \textit{message broker}, e este será responsável por enviar a mensagem ao destinatário quando ele estiver ativo.

As mensagens que chegam ao MOM são armazenados em filas. Por padrão as filas são PEPS (Primeiro a Entrar, Primeiro a Sair) ou FIFO em inglês (\textit{First-IN, First-OUT}). Assim como a ordem, o nome e o tamanho da fila podem ser alterados no sistema de mensagens \cite{Curry2004}.  

Os modelos de mensagem disponíveis no MOM são:
\begin{itemize}
    \item \textbf{Ponto-a-ponto}: As mensagens enviadas ao \textit{message broker} podem ser consumidas por apenas um e somente um serviço.
    \item \textbf{\textit{Publish/Subscribe}}: Uma mensagem enviada ao \textit{message broker} pode ser consumida por vários serviços.
\end{itemize}

\cite{richter:2017} utilizou uma implementação de MOM, \textit{RabbitMQ}, no desenvolvimento de um sistema para compra e venda de passagens de trens na Alemanha. Dentre os requisitos não funcionais estavam alta disponibilidade, ambiente não confiável. A replicação dos microserviços em diversos servidores físicos espalhados, permitiu a redundância e alta disponibilidade do sistema. E a comunicação entre os microserviços foi feita através uma fila de mensagens assíncrona, \textit{RabbitMQ}, permitindo o desacoplamento da comunicação e colaborando na tolerância a falhas também.

Já \cite{Brilhante:2017}, utiliza uma implementação do MOM, \textit{RabbitMQ} também, para construir uma arquitetura de microserviços reativa. Segundo o manifesto reativo (\cite{reactivemanifesto}), um sistema reativo deve ser responsivo, resiliente, elástico e orientado a mensagens. Para garantir as propriedades de um sistema reativo, \cite{Brilhante:2017} utilizou uma coreografia interna, baseada em filas assíncronas, para desacoplar as comunicações diretas de cada microserviço.  

\cite{Abdelouahed:2018} propôs um sistema remoto de gerenciamento de irrigação alimentado por energia solar que utiliza o Vert.x, que contém uma implementação de MOM e permite criar uma fila de mensagens assíncrona com os modelos de mensagem ponto-a-ponto e/ou \textit{publish/subscribe}. Nos sistemas de agricultura ciber-físicos, sensores são utilizados para coletar informações sobre o clima, solo e plantação. A partir destas informações, as decisões são tomadas para gerenciar a irrigação, umidade do solo e verificar a saúde das plantas etc. Tanto as informações do solo, plantação quanto as tomadas de decisão por parte do agricultor são enviados ao \textit{message broker}, que armazena estas mensagens e repassa aos microserviços interessados. Neste caso, \cite{Abdelouahed:2018} utilizou o Vert.x para desenvolver uma arquitetura de microserviços reativa.

\section{Resultados}
\label{sec:resultados}

Conforme a seção \ref{sec:trabalhos-relacionados}, a estratégia mais utilizada para a coreografia dos microserviços foi a colaboração baseada em eventos. A colaboração baseada em eventos é altamente desacoplada. O serviço que emite o evento não sabe quem ou quando alguém irá reagir a tal ação. Isto permite acrescentar mais assinantes sem o emissor do evento saber, por exemplo \cite{Newman:15}.

De acordo com a pesquisa exploratória, com os artigos encontrados, a forma mais comum para os microserviços emitirem e consumirem os eventos é através da utilização de um MOM (\textit{Message Oriented-Middleware}). \cite{Newman:15, Brilhante:2017, richter:2017} citam \textit{RabbitMQ} como uma implementação de MOM a ser utilizada. \textit{RabbitMQ} suporta o protocolo \emph{AMQP} (\textit{Advanced Message Queuing Protocol }). Segundo \cite{Vinoski:06}, AMQP é um protoco de mensagem assíncrona inter-operável. Ele garante que tanto o cliente quanto o MOM, caso tenham suporte ao AMQP, possam trocar mensagens independente da plataforma ou linguagens desenvolvidas.

\cite{Abdelouahed:2018} utilizou o \textit{Vert.x} para criar a arquitetura de microserviços reativa. O \textit{Vert.x} possui um \textit{Event Bus}, que funciona como o \textit{message broker}, recebendo e enviando mensagens aos serviços conectados a ele. Por padrão, o \textit{Event Bus} do \textit{Vert.x} não possui suporte ao protocolo AMQP. Para ter suporte, é necessário utilizar o componente \textit{AMQP Bridge}, que deve ser indicado no arquivo de configuração de compilação.

Inicialmente, \textit{RabbitMQ} parece ser uma solução viável como MOM, pois tem suporte ao protocolo AMQP, que permite a interoperabilidade entre sistemas de diveras plataformas e linguagens. Há outras implementações de MOM que não foram levadas em consideração neste estudo. Por exemplo, Apache ActiveMQ, Amazon SQS, RocketMQ. Ou Kafka e o serviço do Google de \textit{publisher/subscriber} \textit{Pub/Sub}. Um estudo mais amplo poderia contemplar estas ferramentas e verificar quais são as vantagens e desvantagens de cada uma ao adotá-la.

\subsection{Ameaças à validade}

O termo microserviço é muito recente. Segundo \cite{Dragoni2017}, a primeira referência ao termo data de 2011. \cite{Newman:15} definiu coreografia e orquestração como dois possíveis estilos arquiteturais para fazer a integração de microserviços. \cite{Newman:15} apropriou-se do conceito de orquestração e coreografia já conhecidos na arquitetura SOA, em \textit{Web Services} \cite{Peltz:2003}. 

Para realizar a pesquisa exploratória, foi difícil encontrar referências ao termo coreografia ou, em inglês, \textit{choreography}. Foi necessário alterar a \textit{string de busca} e ampliá-la. Utilizou-se os termos \textit{architecture} e \textit{microservices} nas bases da Scopus, IEEE e ACM. E junto com este termo foram filtrados artigos que continham referências a: \textit{event-driven}, \textit{reactive}, \textit{queue}, \textit{pub/sub}.

Dado esta dificuldade, foi necessário recorrer a literatura cinzenta \cite{botelho:2017} para aprimorar o conhecimento sobre coreografia em microserviços. Utilizamo-nos de artigos técnicos, postagem em blogs, \textit{white-papers} como referência. 

Portanto, os artigos discutidos na seção \ref{sec:trabalhos-relacionados} permeiam apenas um conjunto possível de estratégias de coreografia e arquiteturas de microserviços. Esta pesquisa exploratória permitiu-nos familiarizar com as estratégias de coreografia, porém a não utilização de uma revisão sistemática e nem revisão da literatura multivocal, que inclui literatura cinzenta \cite{GAROUSI2018}, representam uma ameaça a validade da pesquisa.

\section{Conclusão}
\label{sec:conclusao}




\bibliographystyle{sbc}
\bibliography{sbc-template}

\end{document}
